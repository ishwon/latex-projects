\documentclass[aspectratio=169]{beamer}
\usepackage{amsmath}
\usepackage{graphicx}
\usetheme{metropolis}
\title{Developers Conference 2018}

\subtitle{Introduction to {\LaTeX}}

\author{Ish Sookun}
\institute[La Sentinelle Ltd] % (optional, but mostly needed)
{
  %\inst{}%
  Linux System Administrator\\
  La Sentinelle Ltd\\
  LSL Digital \textit{(we develop \& host digital products for La Sentinelle Ltd \& subsidiaries)}\\ 
  %openSUSE Advocate (since 2009)
  \begin{flushright}
    \includegraphics[scale=0.15]{logo-lsl.png}
  \end{flushright}
}
\date{19 May 2018}
\subject{Developers Conference 2018}
\setbeamertemplate{footline}{
  % below is the line for 16:9 aspect ratio
  \raisebox{3pt}{\makebox[\paperwidth]{\hfill\makebox[575pt]{Developers Conference 2018 | Introduction to {\LaTeX} | Ish Sookun | Voil\`a Hotel 19 May 2018\hspace{46ex} We're on slide \insertframenumber\hspace{1ex}of \inserttotalframenumber\hspace{1ex}:)}}}}
  % below is the footline for 4:3 aspect ratio
  %\raisebox{3pt}{\makebox[\paperwidth]{\hfill\makebox[500pt]{Developers Conference 2018 | Introduction to {\LaTeX} | Ish Sookun | Voil\`a Hotel 19 May 2018\hspace{10ex} We're on slide \insertframenumber\hspace{1ex}of \inserttotalframenumber\hspace{1ex}:)}}}}
\begin{document}

\begin{frame}[plain,noframenumbering]
  \titlepage
\end{frame}

\begin{frame}{Outline}
  \tableofcontents
  % [pausesections]
\end{frame}

\section{What is {\LaTeX}?}

\subsection{How is it pronounced?}
\begin{frame}{How is it pronounced?}
  There is no specific recommendation on pronouncing {\LaTeX} but the {\TeX} part is pronounced \underline{\bf tech} as explained by its creator Prof. Donald E. Knuth. {\TeX} in Greek means \textit{art}, \textit{craft} or \textit{technique}.

\end{frame}

\subsection{A brief history of {\LaTeX}}
\begin{frame}{A brief history of {\LaTeX}}
  The {\LaTeX} program is a computer typesetting for placing text on a page. It was written by Dr. Leslie Lamport while using {\TeX}'s typesetting engine \& macro system developed by Prof. Donald E. Knuth.

  \hspace{4ex}1962 - {\TeX} created by Prof. Donald E. Knuth\\
  \hspace{4ex}1968 - {\TeX} and Metafont\\
  \hspace{4ex}1982 - {\TeX}82\\
  \hspace{4ex}1983 - {\LaTeX}2.09 created by Dr. Leslie Lamport\\
  \hspace{4ex}1994 - {\LaTeXe}

  {\LaTeX}3 is under development since 2014.
\end{frame}

\subsection{Where is it used?}
\begin{frame}{Where is it used?}
  Over the years {\LaTeX} has become the standard document system for writing technical and scientific papers. It's widely used by the scientific community to produce documents that contains formulas, for example the Schrodinger equation:

  \begin{equation*}
    e^{i\hat{k}.\hat{r}}=\sum\limits_{l=0}^{\infty }\left( 2l+1\right)
    i^{l}j_{l}\left( kr\right) P_{l}\left( \cos \theta \right) 
  \end{equation*}

  {\LaTeX} is also used by scholars to write documents that contain non-Latin scripts such as Arabic, Greek, Sanskrit, Tamil, etc.

\end{frame}

\section{Getting {\LaTeX}}
\subsection{Installation}
\begin{frame}{Installation}
  I use Texmaker on openSUSE Tumbleweed and installation is straighforward via the command-line console:\\

    {\tt \vspace{0.5em}\hspace{8ex}sudo zypper in texmaker}
  
  Installation packages are available for Linux distributions, Windows and macOS from the project website {\bf xm1math.net/texmaker}.\\
  
  {\vspace{1em} Texmaker is a cross-platform open source {\LaTeX} editor with an integrated PDF viewer. It's written in C++ and uses the Qt interface.}
\end{frame}

\subsection{Configuring Visual Studio Code}
\begin{frame}{Configuring Visual Studio Code}
  A {\LaTeX} extension by James Yu is available through the Visual Studio Marketplace. Press Ctrl + P, paste the following command and press enter.
  \begin{center}
    {\tt ext install James-Yu.latex-workshop}
  \end{center}

  The default config will use the {\tt pdflatex} engine but it can be easily configured to use {\tt xelatex}, {\tt lualatex} or some other engine.
\end{frame}

\section{Structure of {\LaTeX} documents}
%\subsection{Example code}
\begin{frame}[fragile]\frametitle{Example code}
  \begin{verbatim}
    % Simple LaTeX document
    
    \documentclass[11pt]{article}
      \title{A Simple {\LaTeX} Document}
      \author{Ish Sookun\\ La Sentinelle Ltd}
      \date{\today}

      \begin{document}
        \begin{abstract}
          This is an abstract.
        \end{abstract}
  \end{verbatim}
\end{frame}

\subsection{documentclass, options, sections etc}
\begin{frame}[fragile]\frametitle{\textbackslash documentclass}
  \begin{columns}[t,onlytextwidth]
    \begin{column}{0.475\textwidth}
       Define the document type using the {\bf documentclass} command. Possible classes are:\\
       \begin{itemize}
         \item {article}
         \item {report}
         \item {book}
         \item {beamer}
       \end{itemize}
    \end{column}
    \begin{column}{0.475\textwidth}  %%<--- here
      The {\bf documentclass} calls a document template that determines document characteristics such as headers, margins, and so on.\\ Example usage:\\ 
      \begin{verbatim}
\documentclass[11pt]{article}
      \end{verbatim}
    \end{column}
  \end{columns}
\end{frame}

\begin{frame}[fragile]\frametitle{\textbackslash begin\{document\}}
  \begin{columns}[t,onlytextwidth]
    \begin{column}{0.475\textwidth}
      Everything that makes up your document should reside within the {\bf document} environment.
      \begin{verbatim}
\begin{document}
  ...
\end{document}

      \end{verbatim} 

    \end{column}
    \begin{column}{0.475\textwidth}
      Within the {\bf document} block you can now add an abstract and sections containing paragraphs.
      \begin{verbatim}
      \section{Introduction}
        All our digital products
        are deployed on Amazon
        Web Services.\\

        Expenses incurred by
        running instances...
      \end{verbatim} 
    \end{column}
  \end{columns}
\end{frame}

\begin{frame}[fragile]\frametitle{\textbackslash maketitle}
  \begin{columns}[t,onlytextwidth]
    \begin{column}{0.475\textwidth}
       To display a title you need to declare the following components in the preamble:\\
       \begin{itemize}
         \item {title}
         \item {author}
         \item {date}
       \end{itemize}
    \end{column}
    \begin{column}{0.475\textwidth}  %%<--- here
      You can accomplish the whole with the following commands:
      \begin{verbatim}
\title{Cloud Expenses Report}
\author{Ish Sookun}
\date{\today}
...
\maketitle
      \end{verbatim}
    These commands should be in the preamble, except the {\bf maketitle} command which comes after the {\tt \textbackslash begin\{document\}}.
    \end{column}
  \end{columns}
\end{frame}

\begin{frame}[fragile]\frametitle{\textbackslash begin\{abstract\}}
  \begin{columns}[t,onlytextwidth]
    \begin{column}{0.475\textwidth}
       A block of text can be formatted by declaring an environment called {\bf abstract}. Environments are delimited by an opening {\tt \textbackslash begin} and a closing {\tt \textbackslash end} tag in the form:
       \begin{verbatim}
\begin{abstract}
This report was prepared after
analyzing the expenses incurred
over a period of twelve months
starting January 2017.
\end{abstract}
       \end{verbatim}       
    \end{column}
    \begin{column}{0.475\textwidth} 
      An abstract is optional and I would recommend it only if you are writing a report that is several pages long. Otherwise, you may also have a {\bf section} that is called {\em abstract} in your report.
    \end{column}
  \end{columns}
\end{frame}

\subsection{Demo}
\begin{frame}[plain]{}
  Woohoo! It's demo time :)
\end{frame}

\begin{frame}{openSUSE Leap 15 - release is coming! 25 May 2018 :)}
  \begin{center}
    \includegraphics[scale=1.0]{official-logo-opensuse.png}
  \end{center}
\end{frame}

\begin{frame}
  \frametitle<presentation>{Further Reading}
    
  \begin{thebibliography}{10}
    
  \beamertemplatebookbibitems
  % Start with overview books.

  \bibitem{texbook}
    The {\TeX}book
    \newblock {by Donald E. Knuth}
  \bibitem{latex}
    {\LaTeX} - A Document Preparation System by Dr. Leslie Lamport
    \newblock {by Dr. Leslie Lamport}
  %\beamertemplatearticlebibitems
  % Followed by interesting articles. Keep the list short. 

  %\bibitem{sharelatex}
  %  Lengths in LaTeX online
  %  \newblock {https://www.sharelatex.com/learn/lengths\_in\_LaTeX}
  \end{thebibliography}

  {\em\small I used the Metropolis beamer theme by Matthias Vogelgesang for this presentation. Its source can be obtained at github.com/matze/mtheme.}
\end{frame}

\end{document}

